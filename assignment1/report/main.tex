\title{CSSE7610 Assignment 1}
\author{Roy Portas - 43560846}
\date{\today}

\documentclass[12pt,a4paper]{article}
\usepackage{amssymb}

\begin{document}

\subsection*{\textbf{Lemma 1}: $q2 \Longrightarrow in \neq out$}
\begin{itemize}
    \item Initially $q2$ is false, lemma is true
    \item Only statement that progresses to $q2$ is $q1$ which requires $in \neq out$
    \item $in \neq out$ cannot become false between $q1$ and $q2$
    \begin{itemize}
        \item Only other statement which can change $in$ or $out$ is $p4$
        \item Since \textbf{lemma 2}, $out \neq (in + 1)\ mod\ N$, so $p4$ cannot increment such that $out = in$ 
    \end{itemize}
\end{itemize}

\subsection*{\textbf{Lemma 2}: $p3..4 \Longrightarrow out \neq (in+1)\ mod\ N$}
\begin{itemize}
    \item Initially holds, as $p3..4$ is false
    \item Only statement that progresses to $p3..4$ is $p2$ which requires $out \neq (in+1)$
    \begin{itemize}
        \item Only statement which can change $in$ or $out$ is $q3$ ($out = out + 1$)
        \item Since \textbf{lemma 1}, $out$ cannot be incremented such that $out = in+1$,
            as this implies that $out = in$
    \end{itemize}
    \item Thus can increment to $out+1$, so $(out+1) \neq (in+1)\ mod\ N$
    \item $out \neq (in + 1)\ mod\ N$ cannot become false between $p2..p4$
    \item Thus cannot increment $in$ such that $in = out$
\end{itemize}

\section{Proof of Mutual Exclusion}

\subsection*{\textbf{Theorem 1}: $\sim(p3 \wedge q2)$}
\begin{itemize}
    \item Assume $p3 \wedge q2$
    \item Using \textbf{lemma 1}: $q2 \Longrightarrow in \neq out \longrightarrow \sim q2 \vee in \neq out$ (Negation of implication)
    \item Using \textbf{lemma 2}: $p3 \Longrightarrow out \neq (in + 1)\ mod\ N \longrightarrow \sim p3 \vee out \neq (in + 1)\ mod\ N$ (Negation of implication)
    \item $(\sim q2 \vee in \neq out) \wedge (\sim p3 \vee out \neq (in + 1)\ mod\ N$
    \item If we assume $p3 \wedge q2$, then $\sim q2 = false$ and $\sim p3 = false$
    \item $(in \neq out) \wedge (out \neq (in + 1)\ mod\ N)$
    \item $in \neq (in + 1)\ mod\ N$
    \item Proof by contradiction
    \item Therefore theorem holds
\end{itemize}

\section{Proof of Freedom from Starvation}

\subsection*{\textbf{Theorem 2}: $\square(p1 \Longrightarrow \lozenge p3) \wedge \square(q1 \Longrightarrow \lozenge q2)$}

\subsubsection*{$\square (p1 \Longrightarrow \lozenge p3)$}
\begin{itemize}
    \item From $p1$, progresses to $p2$
    \item To progress to $p3$, $out \neq (in+1)\ mod\ N$ must be true, using \textbf{Lemma 2}
    \begin{itemize}
        \item Since $p4$ is the only line that can change $in$, therefore the only variable that can update and break the await condition is $out$
        \item The only line which can update $out$ is $q3$
    \end{itemize}
    \item Initially $in = out = 0$, so $0 \neq (0+1)\ mod\ N$ is true, thus progresses to $p3$
    \item For every subsequent run, the $q$ process must run at least once (such that there is something in the buffer to read)
\end{itemize}

\subsubsection*{$\square (q1 \Longrightarrow \lozenge q2)$}
\begin{itemize}
    \item To progress to $q2$, $in \neq out$ must be true, using \textbf{Lemma 1}
    \item Initially $in = out = 0$, thus the process will block until $p$ has been run at least once
    \item Once $p$ has been run at least once, it will increment $in$, making $in \neq out$ true and allowing $q1$ to progress
        \begin{itemize}
            \item Since $q3$ is the only line which can change $out$, therefore the only variable that can update and break the await condition is $in$
            \item The only line that changes $in$ is $p4$, thus $p$ must enter it's critical section before $q$ can run
        \end{itemize}
    \item When an item is added, $in \neq out$ will be true and will progress to $q2$
    \item If $in = out$, the process will block until an item is added, which will increment $in$
\end{itemize}

\end{document}
