\title{CSSE7610 Assignment 1}
\author{Roy Portas - 43560846}
\date{\today}

\documentclass[12pt,a4paper]{article}
\usepackage{amssymb}

\begin{document}

\section{Proof of Mutual Exclusion}

\subsection*{\textbf{Lemma 1}: $q2 \Longrightarrow in \neq out$}
\begin{itemize}
    \item Initially $q2$ is false, lemma is true
    \item Only statement that progresses to $q2$ is $q1$ which requires $in \neq out$
    \item $in \neq out$ cannot become false between $q1$ and $q2$
    \begin{itemize}
        \item Only other statement which can change $in$ or $out$ is $p4$
        \item Since \textbf{lemma 2}, $p$ cannot make $in \neq out$
    \end{itemize}
\end{itemize}

\subsection*{\textbf{Lemma 2}: $p3..4 \Longrightarrow out \neq (in+1)\ mod\ N$}
\begin{itemize}
    \item Initially holds, as $p3..4$ is false
    \item Only statement that progresses to $p3..4$ is $p2$ which requires $out \neq (in+1)$
    \item $out != (in + 1) mod N$ cannot become false between $p2..p4$
    \item Thus cannot increment $in$ such that $in = out$
    \begin{itemize}
        \item Only statement which can change $in$ or $out$ is $q3$ ($out = out + 1$)
        \item Thus can increment to $out+1$, so $(out+1) \neq (in+1)\ mod\ N$
    \end{itemize}
\end{itemize}

\subsection*{\textbf{Theorem 1}: $\sim(p3 \wedge q2)$}
\begin{itemize}
    \item Assume $p3 \hat q2$
    \item $q2 \Longrightarrow in != out \longrightarrow \sim q2 \vee in \neq out$
    \item $p3 \Longrightarrow out \neq (in + 1)\ mod\ N$
    \item $(\neq q2 \vee in \neq out) \wedge (\sim p3 \vee out \neq (in + 1)\ mod\ N$
    \item If we assume $p3 \wedge q2$, then $\sim q2 = false$ and $\sim p3 = false$
    \item $(in \neq out) \wedge (out \neq (in + 1)\ mod\ N)$
    \item $in \neq (in + 1)\ mod\ N$
    \item Therefore theorem holds
\end{itemize}

\section{Proof of Freedom from Starvation}

\subsection*{\textbf{Theorem 2}: $\square(p1 \Longrightarrow \lozenge p3) \wedge \square(q1 \Longrightarrow \lozenge q2)$}

\subsubsection*{$\square (p1 \Longrightarrow \lozenge p3)$}
\begin{itemize}
    \item From $p1$, progresses to $p2$
    \item To progress to $p3$, $out \neq (in+1)\ mod\ N$ must be true
    \item Intially $in = out = 0$, so $0 \neq (0+1)\ mod\ N$ is true, thus progresses to $p3$
\end{itemize}

\subsubsection*{$\square (q1 \Longrightarrow \lozenge q2)$}
\begin{itemize}
    \item To progress to $q2$, $in \neq out$ must be true
    \item When an item is added, $in \neq out$ will be true and will progress to $q2$

\end{itemize}

\end{document}
