\title{CSSE7610 Assignment 1}
\author{Roy Portas - 43560846}
\date{\today}

\documentclass[12pt,a4paper]{article}

\begin{document}

\section{Proof}

\subsection*{\textbf{Lemma 1}: $q2 \Longrightarrow in \neq out$}
\begin{itemize}
    \item Initially $q2$ is false, lemma is true
    \item Only statement that progresses to $q2$ is $q1$ which requires $in \neq out$
    \item $in \neq out$ cannot become false between $q1$ and $q2$
    \begin{itemize}
        \item Only other statement which can change $in$ or $out$ is $p4$
        \item Since \textbf{lemma 2}, $p$ cannot make $in \neq out$
    \end{itemize}
\end{itemize}

\subsection*{\textbf{Lemma 2}: $p3..4 \Longrightarrow out \neq (in+1)\ mod\ N$}
\begin{itemize}
    \item Initially holds, as $p3..4$ is false
    \item Only statement that progresses to $p3..4$ is $p2$ which requires $out \neq (in+1)$
    \item $out != (in + 1) mod N$ cannot become false between $p2..p4$
    \item Thus cannot increment $in$ such that $in = out$
    \begin{itemize}
        \item Only statement which can change $in$ or $out$ is $q3$ ($out = out + 1$)
        \item Thus can increment to $out+1$, so $(out+1) \neq (in+1)\ mod\ N$
    \end{itemize}
\end{itemize}


\end{document}
